\subsection{The component section}\label{the-component-section}

\subsubsection{Extensiveness}\label{extensiveness}

CS1: Provide extensive metadata with regard to the profile

{[}priority: medium{]}

In general, it should be aimed at providing `complete' metadata
information with regard to the selected profile, i.e.~the information
asked for by the profile's components, elements etc. should be provided
as exhaustively as possible. In case optional elements of the selected
profile are not assigned values, those elements should be left out of
the metadata instance, i.e.~empty elements without metadata information
either as element or attribute values should be avoided.

\subsubsection{Resource proxy
references}\label{resource-proxy-references}

CS2: refer to a ResourceProxy for resource specific metadata components

{[}priority: medium{]}

It is possible to refer to the resource a Components section
specifically applies to by including the respective Resource Proxy id in
the \texttt{@cmd:ref} attribute of the component's root element. Don't
refer back to all Resource Proxies, which is possible with CMDI 1.1, as
this is the implicit default.

\subsubsection{Text Elements and Attributes}\label{text-elements-attributes}

CS3: Explicitly name the languages used

{[}priority: medium{]} {[}\emph{TODO: check: CMDI Instance Validator}{]}

\begin{longtable}[c]{@{}l@{}}
\toprule
Note by Menzo\tabularnewline
\midrule
\endhead
High priority?\tabularnewline
\bottomrule
\end{longtable}

The default language for element contents is English, meaning that
English wording should be provided for each element of the CMDI profile
used. Additionally, other languages may be used, complementing the
English version of a CMDI record. The possibility of multilinguality has
to be already included in the design of the component. For the
specification of the language applied, each element should be provided
with an \texttt{@xml:lang} attribute (even when the content is in
English). Use an \href{https://tools.ietf.org/rfc/bcp/bcp47.txt}{BCP 47}
\href{https://tools.ietf.org/rfc/bcp/bcp47.txt}{language tag} to
unambiguously identify the respective language within
\texttt{@xml:lang}.

CS4: Use specific rather than (only) generic metadata

{[}priority: medium{]}

\begin{longtable}[c]{@{}l@{}}
\toprule
\begin{minipage}[b]{0.07\columnwidth}\raggedright\strut
Note by Twan
\strut\end{minipage}\tabularnewline
\midrule
\endhead
\begin{minipage}[t]{0.07\columnwidth}\raggedright\strut
I don't think we have to advise against providing (relevant) generic
metadata as long as the specific information is also there. For example
genere literature + genre crime novel would actually be helpful in terms
of discoverability - unless the value comes from an ontology that allows
us to derive such information, but this generally will not be the case.
\strut\end{minipage}\tabularnewline
\bottomrule
\end{longtable}

The information provided should be specific rather than generic
(e.g.~two items ``\texttt{gesture}'' and ``\texttt{speech}'' for
modality would be preferred to one item ``\texttt{multimodal}'').
Similarly, there should be exactly one value per element. In case of
several suitable values use several elements accordingly (e.g.~prefer
\texttt{\textless{}modality\textgreater{}gesture\textless{}/modality\textgreater{}\textless{}modality\textgreater{}speech\textless{}/modality\textgreater{}}
to
\texttt{\textless{}modality\textgreater{}gesture,\ speech\textless{}/modality\textgreater{}}).
Note: In cases where controlled vocabularies are provided, the values
are governed by those. However, also when metadata creators are free to
type any string, it is best practice to avoid assigning enumerations of
(what is generally perceived as) distinct values to one and the same
element or attribute.

\subsubsection{Vocabularies}\label{vocabularies}

CS5: In open vocabularies make use of the suggested items before
introducing extensions

{[}priority: medium{]} {[}\emph{TODO: partially check: CMDI Instance
Validator}{]}

In case of an open vocabulary provided for an element or attribute the
proposed vocabulary should be applied wherever possible and only
extended in case of gaps with regard to contents/concepts. If the
proposed vocabulary is extended, try to avoid overlapping meanings and
ambiguities among vocabulary items.

\begin{longtable}[c]{@{}l@{}}
\toprule
\begin{minipage}[b]{0.07\columnwidth}\raggedright\strut
Note by Menzo
\strut\end{minipage}\tabularnewline
\midrule
\endhead
\begin{minipage}[t]{0.07\columnwidth}\raggedright\strut
Turn into a separate BP: ``If a suggested value is taken for an element
and a concept link is available in the vocabulary, the link should be
copied into the \texttt{@cmd:ValueConceptLink}''
\strut\end{minipage}\tabularnewline
\bottomrule
\end{longtable}

CS6: Provide consistent vocabulary items

{[}priority: medium{]}

The vocabulary used should be consistent throughout a repository. In
this standardized wording and spelling (e.g.~for named entities such as
organizations, collections, etc.) should be used and abbreviations
should be avoided or provided together with their proper expansion.
Items of the vocabulary used should be mutually exclusive so that they
hence may be unambiguously applied.
